% Klassifiziert den Dokumenten-Typ
% Doku: http://exp1.fkp.physik.tu-darmstadt.de/tuddesign/
% Farben: http://www.tu-darmstadt.de/media/medien_stabsstelle_km/services/medien_cd/das_bild_der_tu_darmstadt.pdf
%  bigchapter: Chapter haben doppelte Schriftgröße
%  linedtoc: Linien im Inhaltsverzeichnis wie bei Überschriften
%  colorbacktitle: Der Dokumenten-Titel wird mir der Accentfarbe hinterlegt
\documentclass[bigchapter,colorback,accentcolor=tud4b,linedtoc,11pt]{tudreport}

% Input Dokument hat das Encoding UTF-8
\usepackage[utf8]{inputenc}
% Wichtiges Paket für Links und verlinktes Inhaltsverzeichnis
\usepackage[ngerman]{hyperref}
% Paket für Fußnoten
\usepackage[stable]{footmisc}
\usepackage{multirow}
\usepackage{colortbl}
% Alternatives package für bilder
\usepackage{wrapfig}
% Paket für amsmath (aligned mathe formeln)
\usepackage{amsmath}
% Farbige tabellen
\usepackage{colortbl}
 
% Paket für Bibliotheks-Verzeichnis, square: Verwende eckige statt runde klammern
% \usepackage[square]{natbib}
% Paket zum Plotten von Datensätzen
\usepackage{pgfplots}
\pgfkeys{%
  /pgfplots/seperators/.style={%
    /pgf/number format/use comma
  },
  /pgfplots/default/.style={%
    /pgf/number format/use comma,
    legend pos=north west,
    x tick label style={/pgf/number format/1000 sep=},
    y tick label style={/pgf/number format/1000 sep=},
    width=0.9\linewidth,
    height=0.40\linewidth,
    scale only axis,
    tick align=outside,
    tickpos=left,
    grid=both,
    tick align=outside,
    minor x tick num=3,
    minor y tick num=4,
    minor grid style={dotted,thin}
  }
}

% Anhänge für Original-Messdaten
\usepackage{fancyvrb}

% Verwende deutsche Bezeichner für Inhaltsverzeichnis, ... (ngerman = New German: neue Rechtschreibung)
\usepackage{ngerman}
% Deutsche Zahlen (entfernt z.B. das Leerzeichen nach einem Dezimal-Komma)
\usepackage{ziffer} 

\usepackage[verbose]{placeins}

%wegen Grafikverschiebung hinzugefügt
\usepackage{float}

%\usepackage{graphicx}
%\usepackage{caption}
\usepackage{subcaption} %Für subfigures

% PDF-Optionen
\hypersetup{%
  pdftitle={TU Darmstadt \- Physikalisches Praktikum für Fortgeschrittene},
  pdfauthor={Esra Bauer, Sören Link},
  pdfsubject={Versuch 4.11},
  pdfview=FitH,
}
% Nummeriere formeln in Subsections einzeln
% Kleines makro zur assymetrischen Fehlerangabe

% Entspricht-Zeichen
\usepackage{scalerel}

\newcommand\equalhat{%
\let\savearraystretch\arraystretch
\renewcommand\arraystretch{0.3}
\begin{array}{c}
\stretchto{
    \scalerel*[\widthof{=}]{\wedge}
    {\rule{1ex}{3ex}}%
}{0.5ex}\\ 
=%
\end{array}
\let\arraystretch\savearraystretch
}
%BEGINN TITELSEITE

\title{Optisches Pumpen}

\subtitle{Esra Bauer  \\Sören Link}

\subsubtitle{Tutor: Simon Mieth \hfill 29.6.2015}

\author{Esra Bauer, Sören Link}

%\settitlepicture{img/title.jpg}

\institution{Physikalisches Praktikum \\für Fortgeschrittene \\ Versuch 4.11}

\date{\today}
%ENDE TITELSEITE


\begin{document}
%ANFANG DOKUMENT

%Titelseite einfügen
\maketitle

%Inhaltsverzeichnis einfügen
\tableofcontents

%ANFANG INHALT
\chapter{Einleitung}

In diesem Versuch geht es um optisches Pumpen, d.h. Erzeugung einer von der thermischen Besetzungsverteilung (Boltzmann-Verteilung) abweichenden Verteilung an Rubidiumatomen, womit man beispielsweise die sehr feine Zeeman-Aufspaltung der Hyperfeinstrukturniveaus spektroskopieren kann. Wir bestrahlen dazu Rubidiumdampf in einem Glaskolben mit einer Rubidiumlampe. Unter anderem wird auch der Kernspin von Rubidium $^{87}$Rb und $^{85}$Rb ermittelt.

\chapter{Grundlagen}

\section{Optisches Pumpen}

Wir können den Rubidiumdampf als kanonisches Ensemble betrachten, d.h. als System, welches an ein Wärmebad der Temperatur T gekoppelt ist. Die Besetzungsverteilung ist dann von T abhängig und durch die Boltzmann-Statistik gegeben. Für die Zahl $N_j$ der Teilchen, die den Zustand j besetzen, gilt demnach:
$$N_j = N_0 \cdot g_j \cdot e^{- \beta E_j},$$
wobei $\beta = \frac{1}{k_B T}$ die Energienormierung bezeichnet und $g_j$ den Entartungsgrad der Energie $E_j$, also die Anzahl der Zustände gleicher Energie $E_j$.

Optisches Pumpen heißt, eine von dieser Statistik abweichende Besetzungsverteilung zu erzeugen. Um dies zu veranschaulichen, betrachten wir ein Energiesystem, das aus drei Niveaus besteht. C sei das höchste Niveau, B das mittlere und A das untere. Von den beiden Grundniveaus A und B seien optischen Übergänge zu C möglich, nicht aber zwischen A und B. Zu Beginn sind nun A und B gleichbesetzt, der Polarisationsgrad ist also: $P := \frac{N_B-N_A}{N_B+N_A} = \frac{N_B-N_A}{N} = 0$.

Nun wird derartiges Licht eingestrahlt, dass nur der Übergang von A nach C induziert wird, von C können Übergänge nach A oder B stattfinden, jedoch nicht von B nach C. Dies bedeutet, immer mehr Teilchen besetzen das Niveau B und immer weniger das Niveau A, da sie von dort kontunierlich nach C gepumpt werden. Dem entgegen wirkt die sogenannte Relaxation, d.h. der Übergang von B nach A. Hier sind zwar keine optischen Übergange möglich, jedoch können Teilchen durch magnetische Dipolübergänge oder durch Stöße dennoch von B nach A übergehen. Man definiert daher die Relaxationsrate wie folgt: 
$$\frac{d n}{d t} \propto \frac{n}{T_1},~ n := N_B-N_A$$
mit der Relaxationszeit $T_1$. Es lässt sich sinnvollerweise auch eine Pumpzeit finden. Diese ist abhängig von der Intensität und von $N_A$. Für die Zunahme der Besetzung $N_B$ von B gilt:
$$d N_B = - d N_A = -K \cdot I \cdot N_A \cdot d t,$$
wobei K eine Proportionalitätskonstante ist. Daraus ergibt sich:
$$d n = d (N_B-N_A) \propto -K \cdot I \cdot (N-n) \cdot d t,~ N_A = \frac{1}{2} (N-n).$$ 
Die Pumpzeit ist damit $T_P = \frac{1}{K \cdot I}$ und $\frac{d n}{d t} \propto \frac{N-n}{T_P}$.

Pumpen und Relaxation erfolgt gleichzeit, wir finden also folgende Differentialgleichung:
$$\frac{d n}{d t} = \frac{N-n}{T_P} - \frac{n}{T_1}$$
und nach kurzer Zeit stellt sich ein Gleichgewicht ein, so dass gelten muss $\frac{d n}{d t} = 0$, woraus folgt: $n_0 = \frac{N}{1+\frac{T_P}{T_1}}$. Um die DGL zu lösen, setze $\frac{1}{\tau} = \frac{1}{T_P} + \frac{1}{T_1}$. Man erhält folgende Lösungen für $n$ und $N_A$:
$$n = n_0 (1-e^{-\frac{t}{\tau}}),~~~ N_A = \frac{1}{2} (N-n_0 (1-e^{\frac{t}{\tau}})).$$

Beim Rubidiumdampf lässt sich der Erfolg des optischen Pumpens (d.h. der Pumpvorgang muss gegenüber der Relaxation überwiegen) an der Transparenz erkennen. Da das Pumplicht nur den Übergang von A nach C induziert, wird dieses Licht zu Anfang stark absorbiert, da noch viele Teilchen im Niveau A befindlich sind und beim Übergang nach C jeweils Licht absorbieren. Je weniger Teilchen sich in A aufhalten, d.h. je stärker der Pumpvorgang fortschreitet, desto weniger Licht wird also absorbiert, wodurch die Transparenz des Rubidiumdampfes sinkt. 

\section{Energieniveaustruktur von Rubidium}

\section{Spektroskopie der Zeeman-Niveaus nach optischem Pumpen}

\section{Bestimmung des Kernspins}

\chapter{Aufbau und Durchführung}

%\begin{figure}[H] 
 % \centering
  %   \includegraphics[width=0.8\textwidth]{img/aufbau.png}
   %  \caption{Experimental setup. As source material we used $^{22}Na$ and
    %   $^{137}Cs$. In addition we mounted plates of aluminium, cadmium, iron and
     %lead behind the source to measure the backscatter radiation. Finally we have
  %   also measured the background radiation without a radiactive source \cite{Anleitung}.}
 % \label{fig:aufbau}
%\end{figure}

\chapter{Auswertung}

\section{Analysis of the spectrum of $^{137}Cs$}


\section{Fit Values}

%\begin{center}
 % \begin{tabular}{r|r|r|r|r|r|r}
  %   Type of peak & $k$     & $\Delta k$ & $\mu$   & $\Delta \mu$ & $\sigma$ & $\Delta \sigma$ \\ \hline
   %  X-ray        & $17066$ & $597$      & $90,7$  & $0,14$       & $3,69$   & $0,16$          \\ \hline
    % Gamma        & $87052$ & $1285$     & $232,4$ & $0,11$       & $7,21$   & $0,14$          \\
%	\end{tabular}
%\end{center}

\section{Energy spectrum}

\section{Background Radiation}

%\begin{center}
%\begin{figure}[H]
%\begin{tikzpicture}
%\begin{axis}[
%    title={Background Radiation},
%    xlabel=channel,
%   ylabel=counts,
%    height=0.7\textwidth,
%    ymin=0,
%    xmin=0,
%    xmax=300,
%    default
%]
%\addplot[red, only marks, mark=x, mark size=1pt] table[x index=0, y index=1] {data/background.txt};
%\addlegendentry{Background}
%\end{axis}
%\end{tikzpicture}
%\captionof{figure}{}
%\end{figure}
%\end{center}

\chapter{Fazit}

%ENDE INHALT
\cleardoublepage{}
% Eintrag fürs Inhaltsverzeichnis
\newpage
\begin{thebibliography}{100}
  \bibitem{Anleitung} {Experimental Instructions} \bibitem{na22decay} {Semibyte, homepage of physics lab assistent and qualified
      computer scientist Tobias Krähling:
      \url{http://www.semibyte.de/wp/download/graphicslib/physics/termschema_na22.png}
    [CC BY-NC-SA 3.0]}
  \bibitem{cs137decay} {Wikipedia, the free encyclopedia. By Tubas-en [Public
      domain], via Wikimedia Commons: \url{http://upload.wikimedia.org/wikipedia/commons/thumb/3/3e/Cs-137-decay.svg/500px-Cs-137-decay.svg.png}}
\end{thebibliography}
\end{document}

%%% Local Variables:
%%% mode: latex
%%% TeX-master: t
%%% End:
